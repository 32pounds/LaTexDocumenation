\documentclass{report}

\begin{document}

\chapter{Map and Movement}
\textbf{Brett Menzies, Gabriel Giovanini de Souze}

\textbf{diagram:}

\textbf{Description:}
The map subsystem revolves around the ``Map'' class,
 which is composed of a ``Layout'' and a ``Tileset'',
 and aggregates sets of ``Item Position'' and ``Player Position'' classes.
 A ``Renderer'' class requires an instance of a ``Map'';
 Its other connections are outside the scope of this diagram.
 A ``Tileset'' aggregates ``Tiles'' (fixed size map tiles) and 
 ``Sprites'' (Other images drawn on a map, like items and characters) for use by the map and rendering code.
 Each ``Item position'' object represents the position and sprite id of items that have been dropped on the map.
 ``Character position'' acts similarly, except it can be modified by ``input'', which represents 
 networked, npc, and local player inputs for the purposes of this diagram.


\end{document}